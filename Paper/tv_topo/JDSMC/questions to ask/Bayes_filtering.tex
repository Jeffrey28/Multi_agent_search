\documentclass[]{article}
\usepackage{amsmath}
%opening
\title{Asymptotic Property of Bayesian Filters}
\author{Chang Liu}

\begin{document}

\maketitle

Assume there exist a moving target in the field $S$. 
We want to use a sensor network to localize the target.
We can model the target dynamics and the sensor measurement model as follows:
\begin{align}
x_{k+1}&=f(x_k,w_k)\\
y_k&=h(x_k,v_k),
\end{align}
where $x_k$ is the target state at time $k$ and $y_k$ is the sensor measurement. $w_k$ and $v_k$ are random noise.

Assume the placement of sensors enables the unique localization of the target, e.g., three triangularly placed range sensors.
A centralized Bayesian filter can utilize the sensor measurements to estimate the target position, which involves two steps:

\textbf{Prediction Step}
\begin{equation*}
P^i_{pdf}(X_k|\mathbf{z}_{1:k-1})
=\int\limits_{X_{k-1}\in S} P(X_k|X_{k-1})P^i_{pdf}(X_{k-1}|\mathbf{z}_{1:k-1})dX_{k-1}.
\end{equation*}

\textbf{Updating Step}
\begin{equation*}
P^i_{pdf}(X_k|\mathbf{z}_{1:k})
= K_iP^i_{pdf}(X_k|\mathbf{z}_{1:k-1})P(\mathbf{z}_k|X_k).
\end{equation*}
$\mathbf{z}_k$ is the set of measurements from the sensors of time $k$.

At each time, we use a point estimate, e.g., MAP, to estimate the target position:
\begin{equation*}
\hat{x}_k=\arg\max_{X\in S} P^i_{pdf}(X_k|\mathbf{z}_{1:k}).
\end{equation*}

Can we show that $\|\hat{x}_k -x_k\|_2 \overset{P}{\longrightarrow} 0$?
If the MAP does not have this property, can some other point estimate, e.g., the mean value or the median of $P^i_{pdf}(X_k|\mathbf{z}_{1:k})$, have this property? 

This seems to be a parameter identification problem, where the target state is the (time-varying) parameter.

\end{document}
