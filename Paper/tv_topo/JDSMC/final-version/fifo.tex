\section{Full-In-and-Full-Out (\proto) Protocol}\label{sec:\proto}		
	\textcolor{\revcol}{This study proposes a Full-In-and-Full-Out (\proto) protocol for measurement exchange in dynamically changing interaction topologies.
	The use of {\proto} allows us to apply measurement dissemination-based distributed filters to time-varying networks.}
% while avoiding the out-of-sequence measurement issue.}
	Let \small$Y^i_{\K}=\lb \left[x^i_k,z^i_k\right]|k\in \K\rb$\normalsize 
	% $Y^i_{\K^{i,j}_k}=\lb \left[x^{i,j}_t,z^{i,j}_t\right]|t\in \K^{i,j}_k\rb$
	be the set of state-measurement pairs of the $i\thi$ UGV, where $\K$ is an index set of time steps.
	Each UGV contains a communication buffer (CB) and a track list (TL). 
	A CB stores state-measurement pairs 
%	consisting of measurements and the corresponding states 
	of all UGVs:
	\small\begin{equation*}		
		\B^i_k=\left[ Y^1_{\K^{i1}_k},\dots,Y^N_{\K^{iN}_k}\right],
%		\mathbf{z}^{CB,i}_k=\left[ z^1_{k^i_1},\dots,z^N_{k^i_N}\right]
	\end{equation*}\normalsize
%	where $z^j_{k^i_j}$ represents the observation made by ${j\thi}$ UGV at time $k^i_j$. 
	where $\B^i_k$ is the CB of $i\thi$ UGV at time $k$ and $\K^{ij}_k (j\in V)$ is the time index set.
	$Y^j_{\K^{ij}_k}$ represents the set of $j\thi$ UGV's state-measurement pairs of time steps in $\K^{ij}_k$ that are stored in $i\thi$ UGV's CB at time $k$. 
	\textcolor{\revcol}{A UGV's TL stores the information of this UGV's reception of all UGVs' measurements, and is used for trimming old state-measurement pairs in the CB to reduce the communication burden.}
	The details of TL will be introduced in \Cref{subsec:tracklist}.
	Each UGV sends its CB and TL to its {\onbhd} at every time step.
	
	\begin{algorithm}
		\caption{{\proto} Protocol}
		\label{alg:lifo}
		\begin{algorithmic}
			\State \textbf{(1)} Initialization.
			
			\CB: 
			The CB of $i\thi$ UGV is initialized as an empty set at $k=0$:
			\small\begin{equation*}
			{\B^i_0=\left[ Y^1_{\K^{i1}_{0}},\dots,Y^N_{\K^{iN}_{0}}\right], \text{ where } Y^j_{\K^{ij}_0} = \lb \left[ \varnothing,\varnothing\right]\rb.}
			\end{equation*}\normalsize
			
			\TL:
			The TL of $i\thi$ UGV is initialized at $k=0$:
			\small\begin{equation*}
			\mathbf{q}^{ij}_1 = \left[0,\dots,0,1\right],\;\text{i.e. } q^{jl}_1=0, \forall j,l\in \lb 1\dots,N\rb.
			\end{equation*}\normalsize
			
			\State \textbf{(2)} At time $k\,(k\geq 1)$ for $i\thi$ UGV:	
			
			%			\State The interaction topology is represented by $G[k]\in\bar{G}$.
			\State (2.1) Receiving Step.
			
			\CB:	The $i\thi$ UGV receives all CBs of its {\inbhd} $\N^\text{in}_i(G_{k-1})$.
			%			The received CBs are totally $|\mathcal{N}_i(G_{k-1})|$ groups, 
%			corresponding to the $(k-1)\thi$ step CBs.
			The received CB from the $l\thi$ UGV is $\B^l_{k-1}\,(l\in\N^\text{in}_i(G_{k-1}))$.
			%			\small\begin{equation*}
			%				\mathcal{B}^l_{k-1}=\left[Y^1_{\K^{l1}_{k-1}},\dots,Y^N_{\K^{lN}_{k-1}}\right],\; l\in\N^\text{in}_i(G_{k-1})
			%			\end{equation*}\normalsize
			
			\TL: The $i\thi$ UGV receives all TLs of its {\inbhd} $\N^\text{in}_i(G_{k-1})$.
			The received TL from the $l\thi$ UGV is $Q^l_{k-1}\,(l\in\N^\text{in}_i(G_{k-1}))$.
			\newline
			
			\State (2.2) Observation Step.
			
			\CB: The $i\thi$ UGV updates $Y^i_{\K^{ii}_{k}}$ by its own state-measurement pair: %at current step:
			\small\begin{equation*}
			Y^i_{\K^{ii}_{k}} = Y^i_{\K^{ii}_{k-1}} \cup \lb{\left[x^i_k,z^i_k\right]}\rb.
			\end{equation*}\normalsize
			
			\State (2.3) Updating Step.
			
			\CB: The $i\thi$ UGV updates other entries of its own CB, $Y^j_{\K^{ij}_{k}}\,(j\neq i)$, by merging with all received CBs:						
			\small\begin{equation*}
			Y^j_{\K^{ij}_{k}} = Y^j_{\K^{ij}_{k-1}} \cup Y^j_{\K^{lj}_{k-1}},
			\; \forall j\neq i,\;\forall l\in \N^\text{in}_i(G_{k-1}).
			\end{equation*}\normalsize
			
			\TL: The $i\thi$ UGV updates its own TL, $Q^i_k$, using the received TLs (see \Cref{alg:upd_tl}).
			\textcolor{\revcol}{The CB is trimmed based on the updated track list (see \Cref{alg:tracklist})}. 
			
			\State (2.4) Sending Step.
			
			\CB: The $i\thi$ UGV sends its updated CB, \small$\B^i_k$\normalsize, 
			%			\small$\B^i_k=\left[Y^1_{\K^{i1}_{k}},\dots,Y^N_{\K^{iN}_{k}}\right]$\normalsize, 
			to all of its {\onbhd} defined in $\N^\text{out}_i(G_k)$.
			
			\TL: The $i\thi$ UGV sends its updated track list, $Q^i_k$, to its {\onbhd} $\N^\text{out}_i(G_k)$.
			
			\State \textbf{(3)} $k\leftarrow k+1$ until stop
		\end{algorithmic}
	\end{algorithm}
	
	The \textbf{{\proto} protocol} is stated in \Cref{alg:lifo},
	\textcolor{\revcol}{which consists of the CB and TL parts. For the purpose of clarity, we focus on the CB parts in this section and leave the description of the TL parts to \Cref{subsec:tracklist}.}		
	\cref{fig:\proto} illustrates the {\proto} cycles in a network of three UGVs with dynamically changing topologies.
	The following facts can be observed from \cref{fig:\proto}: 
	(1) the topologies are jointly strongly connected in the time interval $\left[0,6\right)$;
	(2) each UGV can receive the state-measurement pairs of other UGVs within finite steps.
	Extending these facts to a network of $N$ UGVs, we have the following properties of \proto:
	%\medskip
	
	\begin{thm}\label{prop1}
		Consider a network of $N$ UGVs with dynamically changing interaction topologies \small$\mathbb{G}=\lb G_{1},G_{2},G_{3}\dots,\rb$\normalsize.
%		\todohere{may rephrase the following sentence}
		If $\mathbb{G}$ is \fc, i.e.,
		\begin{enumerate}
			\item there exists an infinite sequence of time intervals $\left[k_m,k_{m+1} \right),\,m=1,2,\dots$, starting at $k_1=0$ and are contiguous, nonempty, and uniformly bounded;
			\item the union of graphs across each such interval is jointly strongly connected,
		\end{enumerate}
		then each pair of UGVs can exchange measurements under \proto. 
		In addition, it takes no more than $NT_u$ steps for a UGV to communicate to another one, 
%		the communication distance (in terms of time) between each pair of UGVs is no greater than \small$(N-1)T_u$\normalsize, 
		where \small$T_u=\sup\limits_{m=1,2,\dots}\left( k_{m+1}-k_m\right) T$ \normalsize is the upper bound of interval lengths.
	\end{thm}
	
	\begin{proof}				
		Without loss of generality, we consider the transmission of $\B^i_{t_1}$ from the $i\thi$ UGV to an arbitrary $j\thi$ UGV ($j\in V\setminus \lb i\rb$), where $t_1\in\left[k_1,k_2 \right)$.
		Since each UGV will receive \inbhd' CBs and send the merged CB to its {\onbhd} at the next time step, the $i\thi$ UGV can transmit $\B^i_{t_1}$ to $j\thi$ UGV if and only if a path $\left[l_1,\dots,l_n\right]$ exists, with $l_1=i$, $l_n=j$, $l_2,\dots,l_{n-1}\in V\setminus\lb i,j\rb$, and the edges $(l_s,l_{s+1})$ appears no later than $(l_{s+1},l_{s+2})$, $s=1,\dots,n-2$.
		
		As the union of graphs across the time interval $\left[k_2,k_3 \right)$ is jointly connected, $i\thi$ UGV can directly send $\B^i_{t_1}$ to at least one another UGV at a time instance, i.e., $\exists l_2\in V\setminus \lb i\rb,\, \exists t_2\in \left[k_2,k_3 \right)$ s.t. $l_2\in\N^\text{out}_{i}(G_{t_2})$.
		If $l_2=j$, then $\B^i_{t_1}$ has been sent to $j$.
		If $l_2\neq j$, $\B^i_{t_1}$ has been merged into $\B^{l_2}_{t_2+1}$ and will be sent out in the next time step. 
		
		Using the similar reasoning for time intervals $\left[k_m,k_{m+1} \right),\;,m=3,4,\dots$, it can be shown that all UGVs can receive the state-measurement pairs in $\B^i_{t_1}$ no later by $k_{N+1}$.
		Therefore, the transmission time from an arbitrary UGV to any other UGVs is no greater than \small$NT_u$\normalsize.
		
%		\hfill\qedsymbol
	\end{proof}

	
%	Similar to the definition in \cite{jadbabaie2003coordination}, we define an interaction topology that satisfies the two conditions in \Cref{prop1} as a \textit{{\fc}} network.

%	\medskip
	\begin{cor}\label{cor1}
		For a {\fc} network, each UGV receives the CBs of all other UGVs under {\proto} within finite time. 
%		This implies $\mathcal{Q}_i = \left\lbrace1,\dots,N \right\rbrace \setminus \left\lbrace i\right\rbrace $.	
%		Additionally, the state-measurement pairs of all UGVs are updated every finite period of time.
	\end{cor}
	\begin{proof}
		%In a network of $N$ UGVs, the maximal length of shortest paths is no greater than $N-1$. 
		According to \Cref{prop1},
%		 the transmission delay between an arbitrary pair of UGVs is no greater than \small$(N-1)T_u$\normalsize.
		each UGV is guaranteed to receive $\B^j_t\; (\forall t\ge 0,\, j\in V)$ when \small$k\geq t+NT_u$\normalsize.
		
%		\hfill\qedsymbol
%		In addition, the state-measurement pairs in each gets updated every finite period of time that is no greater than \small$(N-1)T_u$\normalsize.
	\end{proof}
	
	\begin{rem}
		\textcolor{\revcol}{
		\Cref{prop1} and \Cref{cor1} are the consequences of FIFO and the use of the communication buffer (CB). 
		In fact, if we use the traditional methods that each UGV only sends the current sensor measurement to neighboring UGVs without the use of CB, it can happen that two UGVs may never exchange their sensor measurements, even there exists a path connecting them.
		The condition of \fc ness is also crucial for guaranteeing the consistency of the FIFO-based distributed Bayesian filter, as shown in \Cref{sec:consist_proof}.
	}
	\end{rem}

	
	\begin{figure}%[thpb]
		\centering
		\includegraphics[width=0.5\textwidth]{fig_2}
%		{figures/fifo}
		\caption{Example of {\proto} with three UGVs under dynamically changing interaction topologies. The arrows represent a directed communication link between two UGVs. $\varnothing$ denotes the empty set. For the purpose of clarity, we only show measurements, not the states, in the CB.}
%			in the CB in this example.}
		\label{fig:\proto}
		%		\vspace{-1em}]
	\end{figure}		