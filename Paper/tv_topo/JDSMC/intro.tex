\section{INTRODUCTION}
	\todohere{mention the relation between BF and KF.}
	\todohere{mention the OOSM issue.}
	%Section II: Topology + \proto + Proof
	%Section III: Binary sensor + DBF + Proof.
	%Section IV: Simulation
	
	Unmanned ground vehicles (UGV) that operate without on-board operators have been used for many applications that are inconvenient, dangerous, or impossible to human. Distributed estimation using a group of networked UGVs has been applied to collectively infer status of complex environment, such as intruder detection \cite{chamberland2007wireless} and object tracking \cite{wang2003online}. Several techniques have been developed for distributed estimation, including distributed linear Kalman filters (DKF) \cite{2005distributed}, distributed extended Kalman filters \cite{madhavan2004distributed} and distributed particle filters \cite{gu2007distributed}, etc. The most generic filtering scheme is distributed Bayesian filters (DBF), which can be applied for nonlinear systems with arbitrary noise distributions \cite{bandyopadhyay2014distributed,julian2012distributed}.
	This paper focuses on a communication-efficient DBF for networked UGVs.
	
	The interaction topology plays a central role on the design of DBF, of which two types are widely investigated in literature: fusion center (FC) and neighborhood (NB). In the former, local statistics estimated by each agent is transmitted to a single FC, where a global posterior distribution is calculated at each filtering cycle \cite{zuo2006bandwidth,vemula2006target}. In the latter, each agent individually executes distributed estimation and the agreement of local estimates is achieved by certain consensus strategies \cite{jadbabaie2003coordination,ren2005consensus,olfati2007consensus}. In general, the NB-based distributed filters are more suitable in practice since they do not require a fusion center with powerful computation capability and are more robust to changes in network topology and link failures. So far, the NB-based approaches have two mainstream schemes according to the transmitted data among agents, i.e., \textit{statistics dissemination} (SD) and \textit{measurement dissemination} (MD). In the SD scheme, each agent exchanges statistics such as posterior distributions and likelihood functions within neighboring nodes \cite{hlinka2013distributed}. In the MD scheme, instead of exchanging statistics, each agent sends its observations to neighboring nodes. 
	
%	The pioneering work of statistics dissemination scheme can date back to the 80s of last century \todohere{reference}. 
%	Later, \todohere{reference} have considerably advanced the study of this scheme in the field of distributed estimation. 
	Statistics dissemination scheme has gained increasing interest and been widely investigated during last decade.
	Madhavan et al. (2004) presented a distributed extended Kalman filter for nonlinear systems \cite{madhavan2004distributed}. This filter was used to generate local terrain maps by using pose estimates to combine elevation gradient and vision-based depth with environmental features. Olfati-Saber (2005) proposed a distributed linear Kalman filter (DKF) for estimating states of linear systems with Gaussian process and measurement noise \cite{2005distributed}. Each DKF used additional low-pass and band-pass consensus filters to compute the average of weighted measurements and inverse-covariance matrices. 
%	Sheng et al. (2005) proposed a multiple leader-based distributed particle filter with Gaussian Mixer for target tracking \cite{sheng2005distributed}. Sensors are grouped into multiple uncorrelated cliques, in each of which a leader is assigned to perform particle filtering and the particle information is then exchanged among leaders. 
	Gu (2007) proposed a distributed particle filter for Markovian target tracking over an undirected sensor network \cite{gu2007distributed}. Gaussian mixture models (GMM) were adopted to approximate the posterior distribution from weighted particles and the parameters of GMM were exchanged via average consensus filter. 
	Hlinka et al. (2012) proposed a distributed method for computing an approximation of the joint (all-sensors) likelihood function by means of weighted-linear-average consensus algorithm when local likelihood functions belong to the exponential family of distributions \cite{hlinka2012likelihood}. Saptarshi et al. (2014) presented a Bayesian consensus filter that uses logarithmic opinion pool for fusing posterior distributions of the tracked target \cite{bandyopadhyay2014distributed}. Other examples can be found in \cite{julian2012distributed} and \cite{beaudeau2012target}. 
%	Generally, this scheme can be further categorized into two types: leader-based and consensus-based. In the former, statistics is sequentially passed and updated along a path formed by active UGVs, called leaders. Only leaders perform filtering based on its own measurement and received measurements from local neighbors \cite{sheng2005distributed}. In the latter, every UGV diffuses statistics among neighbors, via which global agreement of the statistics is achieved by using consensus protocols \cite{olfati2007consensus,ren2005consensus,jadbabaie2003coordination}. 
	
	
	Despite the popularity of statistics dissemination, exchanging statistics can consume high communication resources. 
	One remedy is to approximate statistics with parametric models, e.g., Gaussian Mixture Model \cite{sheng2005distributed}, which can reduce communication burden to a certain extent. 
	However, such manipulation increases the computation burden of each agent and sacrifices filtering  accuracy due to approximation.
	The measurement dissemination scheme is an alternative solution to address the issue of exchanging statistics. 
	An early work on measurement dissemination was done by Coates et al. (2004), who used adaptive encoding of observations to minimize communication overhead \cite{coates2004distributed}. Ribeiro et al. (2006) exchanged quantized observations along with error-variance limits considering more pragmatic signal models \cite{ribeiro2006bandwidth}.
	A recent work was conducted by Djuric et al. (2011), who proposed to broadcast raw measurements to other agents, and therefore each agent has a complete set of observations of other agents for executing particle filtering \cite{djuric2011non}.  
	A shortcoming of aforementioned works is that their communication topologies are assumed to be a fixed and complete graph that every pair of distinct agents is constantly connected by a unique edge. 
	In many real applications,the  interaction topology may change dynamically due to unreliable links, external disturbances and/or range limits \cite{xiao2008asynchronous}.
%	The communication links between UGVs may be unreliable due to disturbances or communication range limitations. 
%	If the information is being exchanged by direct sensing, the locally visible neighbors of a vehicle will likely change over time. 
	In such cases, dynamically changing topologies can cause random packet loss and variable transmission delay, thus decreasing the performance of distributed estimation, and even leading to inconsistency and non-consensus. 
	Leung et al. (2010) explored a decentralized filter for dynamic robot networks \cite{leung2010decentralized}.
	The algorithm was shown to achieve centralized-equivalent filtering performance in simulations. 
	However, it required the communication of both measurements and statistics, which could incur large communication overhead.
		
	The main contribution of the paper is that we present a measurement dissemination-based distributed Bayesian filtering (DBF) method for a group of networked UGVs with dynamically changing interaction topologies. 
	In our previous work, we have proposed a Latest-In-and-Full-Out (LIFO) protocol for data exchange and developed a LIFO-based DBF. 
	However, it only applies to static target.
	In this work, we introduce the concept of the track list and extend our methods to time-varying topologies.
	The measurement dissemination scheme uses the so-called Full-In-and-Full-Out (\proto) protocol, under which each UGV is only allowed to broadcast observations to its neighbors by using single-hopping.
	Individual Bayesian filter is implemented locally by each UGV after exchanging observations using \proto.
	Under the condition that the union of undirected switching topologies is connected frequently enough, two properties are achieved: (1) {\proto} can disseminate observations over the network within finite time; (2) \proto-based DBF guarantees the consistency of estimation that each individual estimate of target position converges in probability to the true target position as the number of observations tends to infinity. 
	The main benefit of using \proto is on the reduction of communication burden, with the transmission data volume scaling linearly with the size of the UGV network. 
	
	The rest of this paper is organized as follows: 
	the {\proto} protocol for dynamically changing interaction topologies is formulated in \cref{sec:lifo};
	the \proto-based DBF algorithm is described in \cref{sec:\proto-dbf}, where the consistency of estimation is proved;
	simulation results are presented in \cref{sec:sim} and \cref{sec:conclu} concludes the paper.