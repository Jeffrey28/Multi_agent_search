\section{INTRODUCTION}
	
%	Unmanned ground vehicles (UGV) that operate without on-board operators have been used for many applications that are inconvenient, dangerous, or impossible to human. 
	Estimation using a group of networked UGVs has been utilized to collectively measure environment status \cite{hedrick2011tools}, such as intruder detection \cite{chamberland2007wireless}, and signal source seeking \cite{atanasov2015distributed}, and pollution field estimation \cite{madhag2017distributed},
	due to its merits on low cost, high efficiency, and good reliability.
	The widely adopted estimation approaches include the Kalman Filter, extended Kalman filter, and particle filter \cite{thrun2005probabilistic}, and
%	Several techniques have been developed for distributed estimation, including distributed linear Kalman filter (DKF) \cite{2005distributed}, distributed extended Kalman filter \cite{madhavan2004distributed}, and distributed particle filter \cite{gu2007distributed}. 
	the most generic scheme might be the Bayesian filter because of its applicability for nonlinear systems with arbitrary noise distributions \cite{bandyopadhyay2014distributed,julian2012distributed}.
	In fact, a Bayesian filter can be reduced to other methods in certain conditions.
	For example, under the assumption of linearity and Gaussian noise, a Bayesian filter can be reduced to the Kalman filter \cite{chen2003bayesian}, and for general nonlinear systems, a Bayesian filter can be numerically implemented as a particle filter due to the advantage in computation \cite{chen2003bayesian}.
	Because of this generality, this study focuses on its networked variant, which can track targets using local communication between neighboring UGVs. % under dynamically changing interaction topologies.
	
	The interaction topology plays a central role on the design of networked Bayesian filter, of which two types are widely investigated in literature: centralized filters and distributed filters.
	In the former, local statistics estimated by each agent is transmitted to a single fusion center, where a global posterior distribution is calculated at each filtering cycle \cite{zuo2006bandwidth,vemula2006target}. 
	In the latter, each agent individually executes distributed estimation and the agreement of local estimates is achieved by certain consensus strategies \cite{jadbabaie2003coordination,ren2005consensus,olfati2007consensus}.
	In general, the distributed filters are more suitable in practice since they do not require a fusion center with powerful computation capability and are more robust to changes in network topology and link failures. 
	So far, the distributed filters have two mainstream schemes in terms of the transmitted data among agents, i.e., \textit{statistics dissemination} (SD) and \textit{measurement dissemination} (MD). 
	In the SD scheme, each agent exchanges statistics such as posterior distributions and likelihood functions within neighboring nodes \cite{hlinka2013distributed}. 
	In the MD scheme, instead of exchanging statistics, each agent sends measurements to neighboring nodes. 
	
%	The pioneering work of statistics dissemination scheme can date back to the 80s of last century \todohere{reference}. 
%	Later, \todohere{reference} have considerably advanced the study of this scheme in the field of distributed estimation. 
	The statistics dissemination scheme has been widely investigated during the last decade, especially in the field of signal processing, network control, and robotics.
	Madhavan et al. (2004) presented a distributed extended Kalman filter for nonlinear systems \cite{madhavan2004distributed}. 
	This filter was used to generate local terrain maps by using pose estimates to combine elevation gradient and vision-based depth with environmental features. 
	Olfati-Saber (2005) proposed a distributed linear Kalman filter (DKF) for estimating states of linear systems with Gaussian process and measurement noise \cite{2005distributed}. 
	Each DKF used additional low-pass and band-pass consensus filters to compute the average of weighted measurements and inverse-covariance matrices. 
%	Sheng et al. (2005) proposed a multiple leader-based distributed particle filter with Gaussian Mixer for target tracking \cite{sheng2005distributed}. Sensors are grouped into multiple uncorrelated cliques, in each of which a leader is assigned to perform particle filtering and the particle information is then exchanged among leaders. 
	Gu (2007) proposed a distributed particle filter for Markovian target tracking over an undirected sensor network \cite{gu2007distributed}. Gaussian mixture models (GMM) were adopted to approximate the posterior distribution from weighted particles and the parameters of GMM were exchanged via average consensus filter. 
	Hlinka et al. (2012) proposed a distributed method for computing an approximation of the joint (all-sensors) likelihood function by means of weighted-linear-average consensus algorithm when local likelihood functions belong to the exponential family of distributions \cite{hlinka2012likelihood}. Saptarshi et al. (2014) presented a Bayesian consensus filter that uses logarithmic opinion pool for fusing posterior distributions of the tracked target \cite{bandyopadhyay2014distributed}. Other examples can be found in \cite{julian2012distributed} and \cite{beaudeau2012target}. 
%	Generally, this scheme can be further categorized into two types: leader-based and consensus-based. In the former, statistics is sequentially passed and updated along a path formed by active UGVs, called leaders. Only leaders perform filtering based on its own measurement and received measurements from local neighbors \cite{sheng2005distributed}. In the latter, every UGV diffuses statistics among neighbors, via which global agreement of the statistics is achieved by using consensus protocols \cite{olfati2007consensus,ren2005consensus,jadbabaie2003coordination}. 
	
	
	Despite the popularity of statistics dissemination, exchanging statistics can consume high communication resources if the environment to be detected is relatively large in space and complicated in structure. 
	One remedy is to approximate statistics with parametric models, e.g., Gaussian Mixture Model \cite{sheng2005distributed}, which can reduce communication burden to a certain extent. 
	However, such manipulation increases the computation burden of each agent and sacrifices filtering  accuracy due to approximation.
	The measurement dissemination scheme is an alternative solution to address the issue of exchanging large-scale statistics. 
	An early work on measurement dissemination was done by Coates et al. (2004), who used adaptive encoding of observations to minimize communication overhead \cite{coates2004distributed}. Ribeiro et al. (2006) exchanged quantized observations along with error-variance limits considering more pragmatic signal models \cite{ribeiro2006bandwidth}.
	A recent work was conducted by Djuric et al. (2011), who proposed to broadcast raw measurements to other agents, and therefore each agent has a complete set of observations of other agents for executing particle filtering \cite{djuric2011non}.  
	A shortcoming of aforementioned works is that their communication topologies are assumed to be a fixed and complete graph that every pair of distinct agents is constantly connected by a unique edge. 
	In many real applications, the interaction topology can be time-varying due to unreliable links, external disturbances and/or range limits \cite{xiao2008asynchronous}.
%	The communication links between UGVs may be unreliable due to disturbances or communication range limitations. 
%	If the information is being exchanged by direct sensing, the locally visible neighbors of a vehicle will likely change over time. 
	In such cases, dynamically changing topologies can cause random packet loss, variable transmission delay, and out-of-sequence measurement (OOSM) issues \cite{xia2009networked}, thus decreasing the performance of distributed estimation.
%	, and even leading to inconsistency and non-consensus. 
	Leung et al. (2010) has explored a decentralized Bayesian filter for dynamic robot networks \cite{leung2010decentralized} in order to achieve centralized-equivalent filtering performance.
	However, it requires the communication of both measurements and statistics, which can still incur large communication overhead.
		
	The main contribution of the paper is to design a distributed Bayesian filtering (DBF) method only using measurement dissemination for a group of networked UGVs with dynamically changing interaction topologies. 
	In our previous work \cite{liu2016distributed}, we have proposed a Latest-In-and-Full-Out (LIFO) protocol for measurement exchange and developed a corresponding DBF algorithm.
	However, it only applies to static targets with simple binary sensor model because each UGV only transmits the latest measurements in the network, leading to the out-of-sequence measurement problem caused by moving targets.
	In this work, we introduce the a new protocol, called the Full-In-and-Full-Out (\proto), which additionally includes a track list of communication history.
	The {\proto} allows each UGV to broadcast a history of measurements to its neighbors by using single-hopping, thus enabling the tracking of moving targets with more general sensor models under time-varying topologies.
%	The measurement dissemination scheme uses the so-called Full-In-and-Full-Out (\proto) protocol, under which each UGV is only allowed to broadcast observations to its neighbors by using single-hopping.
	An individual Bayesian filter is implemented locally by each UGV after exchanging observations using \proto.
	Under the condition that the topologies is \fc, {\proto} can disseminate measurements over the network within a finite time.
%	two properties are achieved: (1) {\proto} can disseminate observations over the network within finite time; (2) \proto-based DBF guarantees the consistency of estimation that each individual estimate of target position converges in probability to the true target position as the number of observations tends to infinity. 
	The main benefit of using {\proto} in the distributed Bayesian filter is on the reduction of communication burden while avoiding the OOSM issue and ensuring that no information loss occurs.
%	, with the transmission data volume scaling linearly with the size of the UGV network. 
	
	The rest of this paper is organized as follows: 
	\Cref{sec:prob} formulates the target tracking problem using multiple UGVs;
	\Cref{sec:\proto} proposes the {\proto} protocol for measurement communication in dynamically changing interaction topologies;
	\Cref{sec:\proto-dbf} introduces \proto-based DBF algorithm and track list;
%	 where the consistency of estimation is proved;
	simulation results are presented in \cref{sec:sim} and \cref{sec:conclu} concludes the paper.