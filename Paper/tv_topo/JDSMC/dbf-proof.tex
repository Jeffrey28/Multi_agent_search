\section{Proof of Consistency}\label{sec:consist_proof}
	%This section proves consistency and consensus of \proto-DBF. 
	%Only proofs for localizing static target are presented, including static UGVs and moving UGVs.
	%The proof of \proto-DBF for moving target is similar to that of static target by considering the dynamic model of the target, but with more complicated algebraic manipulation. 
	
	%Considering $S$ is finite and $x^{T^*}$ is the true location of target, define an \textit{equivalent-location} set $X^T_{eq}\subseteq S$ such that 
	%\small\begin{equation*}
	%X^T_{eq}=\left\lbrace x^T\in S\arrowvert P(z_k|x^{T})=P(z_k|x^{T^*}),\; \forall z_k\in \left\lbrace 0,1\right\rbrace\right\rbrace ,
	%\end{equation*}\normalsize
	%i.e., $x^T\in X^T_{eq}$ gives the same observation likelihood as $x^{T^*}$ for given UGV positions.
	%%for any two parameters in $X^T_{eq}$, the sensor model with one of these two parameters generates equivalent probability value for the same observation $z_k$
	%Since $S$ is finite, $X^T_{eq}$ is also finite. 
	%%Let $X^T_{eq,1},\dots,X^T_{eq,u}$ denote all equi-parameter sets that partition $X^T$ such that following properties hold:
	%%\begin{enumerate}
	%%	\item $\bigcup^u_{i=1} X^T_{eq,i}= X^T$
	%%	\item $X^T_{eq,i}\cap X^T_{eq,j}=\emptyset,\;i,j\in \left\lbrace 1,\dots,u\right\rbrace ,i\neq j.$
	%%\end{enumerate}
	%%Without loss of generality, assume $x^{T^*}\in X^T_{eq,1}$, where $x^{T^*}$ denotes the actual position of the target. 
	%
	%%\todohere{double check if the following examples for equi-parameter set are accurate}
	%
	%%\begin{rem}
	%%	the equivalent-location set depends on the property of the sensor. 
	%%	For example, for a laser scanner with high-fidelity sensing capability, each equivalent-location set contains only a small number of target positions.
	%	The reason to introduce equivalent-location set is that ghost target might exist in some special UGV arrangement and sensor types.
	%	For example, for undirected binary sensors that are linearly arranged, a ghost target can exist at the mirror position of the true target.
	%	When sensors are overlapped at a single point, ghost targets can exist on a circle that contains the true target.
	%	In theory, DBF cannot rule out ghost targets in such cases and prior knowledge is needed for further clarification.
	%%	Therefore, this study only proves the convergence to the equivalent-location set rather to the true location.
	%	However, by using high-fidelity sensors, such as cameras and laser scanners, and multiple observations from different UGV placements, equi-location set can be reduced to only contain true location of target.
	
	%\end{rem}
	
	This section proves the consistency of the \textit{maximum a posteriori} (MAP) estimator of LIFO-DBF under unbiased sensors (sensors without offset).
	A state estimator is \textit{consistent} if it converges in probability to the true value of the state \cite{amemiya1985advanced}.
	Consistency is an important metric for stochastic filtering approaches \cite{chen2003bayesian}, which 
%	which implies that all UGVs' estimation results converge to a same value, while the consistency
	not only implies achieving consensus asymptotically, but also requires the estimated value converge to the true value.	
	We first prove the consistency for static UGVs and then for moving UGVs. 
%	For simplicity and clarity, 
	Here we assume that $S$ is a finite set (e.g. a finely discretized field) and the target is relatively slow compared to the filtering dynamics.
%	This makes sense if the convergence is achieved much faster than the target's dynamics.
	In addition, the target position can be uniquely determined by the multi-UGV network with proper placement (i.e., excluding the special case of ghost targets \cite{malanowski2012two}).
	
	\subsection{Static UGVs}	
	The consistency of \proto-DBF for static UGVs is stated as follows: 
%	This section proves that \proto-DBF achieves consistent estimation of target position provided that the interaction topologies are jointly connected frequently enough as the system evolves.
%	To be specific, assume that $S$ is finite, the target moves in a deterministic way, and $x^{T^*}$ is the true position of target, then the consistency of \proto-DBF is stated as follows:
	
	\begin{thm}\label{thm:\proto-dbf-sta-ugv}
		Assume the UGVs are static and the sensors are unbiased. If the network of $N$ UGVs is \fc, then the MAP estimator of target position converges in probability to the true position of the target using \proto-DBF, i.e.,
		
		\small\begin{equation*}
		\lim\limits_{k\rightarrow \infty}
		P(\X_k^{MAP}=\xg)=
		1,\;i\in V,
		\end{equation*}\normalsize
		where 
		\small\begin{equation*}
		\X_k^{MAP}=\arg\max\limits_{\X}P^i_{pdf}(\X|\Z^{i}_{1:k}).
		\end{equation*}		
	\end{thm}
%	\medskip
	
	\begin{proof}	
		Define the time set of $i\thi$ UGV, $\K^{i,j}_{k}(\,j\in V)$, that contains the time steps of measurements by the $j\thi$ UGV that are contained in $B^i_k$.
		The batch form of DBF at $k\thi$ step is
		\small\begin{equation*}
		P^i_{pdf}(\X|\Z^i_{1:k})=\frac{P^i_{pdf}(\X)\prod\limits_{j=1}^{N}\prod\limits_{t\in\K^{i,j}_{k}}P(z^j_t|\X)}{\sum\limits_{\X\in S}P^i_{pdf}(\X)\prod\limits_{j=1}^{N}\prod\limits_{t\in\K^{i,j}_{k}}P(z^j_t|\X)}.
		\end{equation*}\normalsize
		
		Comparing $P^i_{pdf}(X_k=x|\Z^i_{1:k})$ with $P^i_{pdf}(X_k=\xg|\Z^i_{1:k})$\footnote{For the purpose of simplicity, we use $P^i_{pdf}(x|\Z^i_{1:k})$ to represent $P^i_{pdf}(X_k=x|\Z^i_{1:k})$ in this proof.} yields
		
		\small\begin{equation}\label{eqn:cmp}
		\frac{P^i_{pdf}(x|\Z^i_{1:k})}{P^i_{pdf}(\xg|\Z^i_{1:k})}=\frac{P^i_{pdf}(x)\prod\limits_{j=1}^{N}\prod\limits_{t\in\K^{i,j}_{k}}P(z^j_t|x)}{P^i_{pdf}(\xg)\prod\limits_{j=1}^{N}\prod\limits_{t\in\K^{i,j}_{k}}P(z^j_t|\xg)}.
		\end{equation}\normalsize
		
		Take the logarithm of \Cref{eqn:cmp} and average it over $k$ steps:
		\small\begin{equation}\label{eqn:cmp_log}
		\frac{1}{k}\ln\frac{P^i_{pdf}(x|\Z^i_{1:k})}{P^i_{pdf}(\xg|\Z^i_{1:k})}=\frac{1}{k}\ln\frac{P^i_{pdf}(x)}{P^i_{pdf}(\xg)}+\sum\limits_{j=1}^{N}\frac{1}{k}\sum\limits_{t\in\K^{i,j}_{k}}\ln\frac{P(z^j_t|x)}{P(z^j_t|\xg)}.
		\end{equation}\normalsize
		
		Since $P^i_{pdf}(x)$ and $P^i_{pdf}(\xg)$ are bounded and nonzero by the choice of the initial PDF, $\lim\limits_{k\rightarrow \infty}\frac{1}{k}\ln\frac{P^i_{pdf}(x)}{P^i_{pdf}(\xg)}= 0.$
		The law of large numbers yields		
		
		\small\begin{subequations}\label{eqn:cmp_lim2}
			\begin{align}
			\frac{1}{k}\sum\limits_{t\in\K^{i,j}_{k}}\ln\frac{P(z^j_t|x)}{P(z^j_t|\xg)}&\overset{P}{\longrightarrow}\mathbb{E}_{z^j_t} \left[\frac{P(z^j_t|x)}{P(z^j_t|\xg)}\right]\\
			&=\int_{z^j_t} P(z^j_t|\xg)\frac{P(z^j_t|x)}{P(z^j_t|\xg)} dz^j_t\\
			&=-D_{KL}\left(P(z^j_t|x)\|P(z^j_t|\xg)\right),
			\end{align}
		\end{subequations}\normalsize		
		where ``$\overset{P}{\longrightarrow}$" represents ``convergence in probability'' and $D_{KL}(P_1\|P_2)$ denotes the Kullback-Leibler (KL) divergence between two probability distribution $P_1$ and $P_2$.
		KL divergence has the property that $\forall P_1,\,P_2, \; D_{KL}(P_1\|P_2)\leq 0$, and the equality holds if and only if $\; P_1=P_2.$ Therefore
%		This leads to the following conclusion:
		%		\begin{subequations}
		\small\begin{align*}
		\lim\limits_{k\rightarrow \infty}\frac{1}{k}\sum\limits_{t\in\K^{i,j}_{k}}\ln\frac{P(z^j_t|x)}{P(z^j_t|\xg)}<0,&\quad x\neq \xg\\
		\lim\limits_{k\rightarrow \infty}\frac{1}{k}\sum\limits_{t\in\K^{i,j}_{k}}\ln\frac{P(z^j_t|x)}{P(z^j_t|\xg)}=0,&\quad x= \xg.
		\end{align*}\normalsize
		%		\end{subequations}
		
		Considering the limiting case of \Cref{eqn:cmp_log}, we get
		%		\begin{subequations}
		\small\begin{align}
		\lim\limits_{k\rightarrow \infty}\frac{1}{k}\ln\frac{P^i_{pdf}(x|\Z^i_{1:k})}{P^i_{pdf}(\xg|\Z^i_{1:k})}<0,&\quad x\neq \xg\label{subeqn:limit1}\\
		\lim\limits_{k\rightarrow \infty}\frac{1}{k}\ln\frac{P^i_{pdf}(x|\Z^i_{1:k})}{P^i_{pdf}(\xg|\Z^i_{1:k})}=0,&\quad x= \xg\label{subeqn:limit2}.
		\end{align}\normalsize
		%		\end{subequations}
		\Cref{subeqn:limit1,subeqn:limit2} imply that
		
		\small\begin{equation*}
		\frac{P^i_{pdf}(x|\Z^i_{1:k})}{P^i_{pdf}(\xg|\Z^i_{1:k})}\overset{P}{\longrightarrow}
		\begin{cases}
		0\quad x\neq \xg,\\
		1\quad x= \xg.
		\end{cases}		
		\end{equation*}\normalsize
		
		Therefore,
		\small\begin{equation*}
		\lim\limits_{k\rightarrow \infty}
		P(X_k^{MAP}=\xg)=1.
		\end{equation*}\normalsize		
		
%		\hfill\qedsymbol
	\end{proof}
	
%%% proof for moving target. seems to be wrong.
%\begin{thm}\label{thm:\proto-dbf-sta-ugv}
%	Assume the UGVs are static and the sensors are unbiased. If the network of $N$ UGVs is \fc, then the MAP estimator of target position converges in probability to the true position of the target using \proto-DBF, i.e.,
%	
%	\small\begin{equation*}
%	\lim\limits_{k\rightarrow \infty}
%	P(\X_k^{MAP}=\xg_k)=
%	1,\;i\in V,
%	\end{equation*}\normalsize
%	where 
%	\small\begin{equation*}
%	\X_k^{MAP}=\arg\max\limits_{\X}P^i_{pdf}(\X_k|\mathbf{z}^{i}_{1:k}).
%	\end{equation*}		
%\end{thm}

%	\begin{proof}	
%		Define the time set of $i\thi$ UGV, $\K^{i,j}_{k}(\,j\in V)$, that contains the time steps of measurements by the $j\thi$ UGV that are contained in $B^i_k$.
%		The batch form of DBF at $k\thi$ step is
%		\small\begin{equation*}
%			P^i_{pdf}(\X_k|\zb^i_{1:k})=\frac{P^i_{pdf}(\X_0)\prod\limits_{j=1}^{N}\prod\limits_{t\in\K^{i,j}_{k}}P(z^j_t|\X_t)P(\X_t|\X_{t-1})}{\sum\limits_{\X_0,\dots,\X_k\in S}P^i_{pdf}(\X_0)\prod\limits_{j=1}^{N}\prod\limits_{t\in\K^{i,j}_{k}}P(z^j_t|\X_t)P(\X_t|\X_{t-1})}.
%		\end{equation*}\normalsize
%
%		Comparing $P^i_{pdf}(X_k=x_k|\zb^i_{1:k})$ with $P^i_{pdf}(X_k=\xg_k|\zb^i_{1:k})$\footnote{For the purpose of simplicity, we use $P^i_{pdf}(x_k|\zb^i_{1:k})$ to represent $P^i_{pdf}(X_k=x_k|\zb^i_{1:k})$ in this proof.} yields
%
%		\small\begin{equation}\label{eqn:cmp}
%			\frac{P^i_{pdf}(x_k|\zb^i_{1:k})}{P^i_{pdf}(\xg_k|\zb^i_{1:k})}=\frac{P^i_{pdf}(x_0)\prod\limits_{j=1}^{N}\prod\limits_{t\in\K^{i,j}_{k}}P(z^j_t|x_t)}{P^i_{pdf}(\xg_0)\prod\limits_{j=1}^{N}\prod\limits_{t\in\K^{i,j}_{k}}P(z^j_t|\xg_t)}.
%		\end{equation}\normalsize
%				
%		Take the logarithm of \Cref{eqn:cmp} and average it over $k$ steps:
%		\small\begin{equation}\label{eqn:cmp_log}
%			\frac{1}{k}\ln\frac{P^i_{pdf}(x_k|\zb^i_{1:k})}{P^i_{pdf}(\xg_k|\zb^i_{1:k})}=\frac{1}{k}\ln\frac{P^i_{pdf}(x_0)}{P^i_{pdf}(\xg_0)}+\sum\limits_{j=1}^{N}\frac{1}{k}\sum\limits_{t\in\K^{i,j}_{k}}\ln\frac{P(z^j_t|x_t)}{P(z^j_t|\xg_t)}.
%		\end{equation}\normalsize
%		
%		Since $P^i_{pdf}(x_0)$ and $P^i_{pdf}(\xg_0)$ are bounded, then $\lim\limits_{k\rightarrow \infty}\frac{1}{k}\ln\frac{P^i_{pdf}(x_0)}{P^i_{pdf}(\xg_0)}= 0.$
%		The law of large numbers yields		
%		
%		\todohere{big mistake: I need to differentiate the moving target and static target. For moving target, the LLN does not apply directly. Gosh, this part seems to be a huge problem. Not sure if it's worth to prove anything in this case...}
%		\small\begin{subequations}\label{eqn:cmp_lim2}
%			\begin{align}
%			\frac{1}{k}\sum\limits_{t\in\K^{i,j}_{k}}\ln\frac{P(z^j_t|x_t)}{P(z^j_t|\xg_t)}&\overset{P}{\longrightarrow}\mathbb{E}_{z^j_t} \left[\frac{P(z^j_t|x_t)}{P(z^j_t|\xg_t)}\right]\\
%			&=\int_{z^j_t} P(z^j_t|\xg_t)\frac{P(z^j_t|x_t)}{P(z^j_t|\xg_t)} dz^j_t\\
%			&=-D_{KL}\left(P(z^j_t|x_t)\|P(z^j_t|\xg_t)\right),
%			\end{align}
%		\end{subequations}\normalsize		
%		where ``$\overset{P}{\longrightarrow}$" represents ``convergence in probability'' and $D_{KL}(P_1\|P_2)$ denotes the Kullback-Leibler (KL) divergence between two probability distribution $P_1$ and $P_2$.
%		KL divergence has the property that $\forall P_1,\,P_2, \; D_{KL}(P_1\|P_2)\leq 0\, \text{and equality holds if and only if } \; P_1=P_2.$
%		This leads to the following conclusion:
%%		\begin{subequations}
%			\small\begin{align*}
%			\lim\limits_{k\rightarrow \infty}\frac{1}{k}\sum\limits_{t\in\K^{i,j}_{k}}\ln\frac{P(z^j_t|x_t)}{P(z^j_t|\xg_t)}<0,&\quad x_t\neq \xg_t\\
%			\lim\limits_{k\rightarrow \infty}\frac{1}{k}\sum\limits_{t\in\K^{i,j}_{k}}\ln\frac{P(z^j_t|x_t)}{P(z^j_t|\xg_t)}=0,&\quad x_t= \xg_t.
%			\end{align*}\normalsize
%%		\end{subequations}
%		
%		Then by considering the limiting case of \Cref{eqn:cmp_log}, we can get:
%%		\begin{subequations}
%			\small\begin{align}
%			\lim\limits_{k\rightarrow \infty}\frac{1}{k}\ln\frac{P^i_{pdf}(x_l|\zb^i_{1:k})}{P^i_{pdf}(\xg_l|\zb^i_{1:k})}<0,&\quad x_l\neq \xg_l\label{subeqn:limit1}\\
%			\lim\limits_{k\rightarrow \infty}\frac{1}{k}\ln\frac{P^i_{pdf}(x_l|\zb^i_{1:k})}{P^i_{pdf}(\xg_l|\zb^i_{1:k})}=0,&\quad x_l= \xg_l\label{subeqn:limit2}.
%			\end{align}\normalsize
%%		\end{subequations}
%		\Cref{subeqn:limit1,subeqn:limit2} imply that
%
%		\small\begin{equation*}
%		\frac{P^i_{pdf}(x_l|\zb^i_{1:k})}{P^i_{pdf}(\xg_l|\zb^i_{1:k})}\overset{P}{\longrightarrow}
%		\begin{cases}
%		0\quad x_l\neq \xg_l,\\
%		1\quad x_l= \xg_l.
%		\end{cases}		
%		\end{equation*}\normalsize
%
%		Therefore,
%		\small\begin{equation*}
%		\lim\limits_{k\rightarrow \infty}
%		P(X_k^{MAP}=\xg_k)=1.
%		\end{equation*}\normalsize		
%	\end{proof}
	
%	\begin{rem}
%		%	When $X^T_{eq}$ only contains $x^{T^{*}}$, consistency means the estimated target position converges to true target position in probability. Additionally, 
%		\Cref{thm:\proto-dbf-sta-tar} guarantees the consistency of distributed filtering using \proto-DBF.
%		This result also leads to the consensus of individual PDFs since they all converge to the same distribution.
%%		This paper can guarantee both consistency and consensus of individual PDF. 
%%		The reason is that all individual PDFs converge to the same distribution, thus the consensus is also achieved.
%		Different from this study, traditional statistics dissemination-based methods only ensure consensus of individual PDFs \cite{bandyopadhyay2014distributed,julian2012distributed}. 
%		To the best knowledge of authors, there is no proof of consistency on estimated target position.
%		%	that guarantees the agreed PDF is close to the true target PDF.
%		%	 of individual PDFs.
%		%	In fact, the statistics dissemination-based methods can ensure the convergence of the state estimate among UGVs. 
%		%	However, there's no guarantee whether the agreed estimate is close to the actual target position.
%	\end{rem}
	
%	\begin{rem}
%%		It is interesting to notice that the interaction topology condition for consistency and consensus in this study is similar to the condition for information consensus in \cite{jadbabaie2003coordination}.
%%		However, the problems studies in this work and in \cite{jadbabaie2003coordination} are different.
%		Different from \cite{jadbabaie2003coordination} that focuses on the information consensus by directly manipulating the communicated individual information among neighboring agents via average-consensus method.
%		This study does not modify the exchanged information themselves during the communication process.
%		Instead, consistency and consensus is achieved as a result of the dissemination of individual observations within the network.		
%	\end{rem}
	
	\subsection{Moving UGVs}
%	This subsection considers the case of using moving UGVs to localize a target, either static or moving.
	The consistency proof for the moving UGVs case is different from the static UGVs case in that each moving UGV makes measurements at multiple different positions.
	We classify UGV measurement positions into two disjoint sets: \textit{infinite-measurement spots} that contain positions where a UGV keeps revisiting as time tends to infinity, and \textit{finite-measurement spots} that contain positions where the UGV visits finitely many times (i.e., the UGV does not visit again after a finite time period).
	It is easy to know that each UGV has at least one position where it revisits infinitely many times as $k$ tends to infinity.
	%, as $k\rightarrow\infty$.
%	Before stating the main theorem, the following lemma is introduced.
%	\begin{lem}\label{lem1}
%		For a set of UGVs moving within a collection of finite positions, each UGV has at least one position where infinite measurements are made as $k$ tends to infinity.
%	\end{lem}
%		
%	\begin{proof}
%		Let $n^{i,k}_s$ denote the times that $i\thi$ UGV visits $s\thi$ position up to time $k$. Then, $\sum\limits_{s\in S} n^{i,k}_s=k$. It is straightforward to see that $\exists n^{i,k}_s,$ such that $n^{i,k}_s\rightarrow \infty,$ as $k\rightarrow \infty$.
%	\end{proof}
%	\medskip
			
	\begin{thm}\label{thm:\proto-dbf-mov-ugv}
		Assume UGVs move within a collection of finite positions and sensors are unbiased, then the MAP estimator of target position converges in probability to the true position of the target using \proto-DBF, i.e.,
		\small\begin{equation*}
		\lim\limits_{k\rightarrow \infty}
		P(\X^{MAP}_k=\xg)=1,\;i\in V.
		%		P(x=\xg|\mathbf{z}^i_{1:k})=1,\;i=1,\dots,N.
		\end{equation*}\normalsize
	\end{thm}
		
	\begin{proof}		
%		The batch form of DBF at $k\thi$ step is
		Similar to \Cref{eqn:cmp}, comparing $P^i_{pdf}(x|\Z^i_k)$ and $P^i_{pdf}(\xg|\Z^i_k)$ yields		
		\small\begin{equation}\label{eqn:cmp2}
		\frac{P^i_{pdf}(x|\Z^i_k)}{P^i_{pdf}(\xg|\Z^i_k)}=\frac{P^i_{pdf}(x)\prod\limits_{j=1}^{N}\prod\limits_{t\in\K^{i,j}_{k}}P(z^j_t|x;x^j_t)}{P^i_{pdf}(\xg)\prod\limits_{j=1}^{N}\prod\limits_{t\in\K^{i,j}_{k}}P(z^j_t|\xg;x^j_t)}.
		\end{equation}\normalsize
			
		The only difference from \Cref{eqn:cmp} is that $P(z^j_l|x;x^j_l)$ in \Cref{eqn:cmp2} varies as the UGV moves.
		%, needs to be grouped according to the UGV position. 
%		For each UGV there exists at least one position where infinite measurements are made as $k\rightarrow \infty$, according to \Cref{lem1}. 
		%All positions can be classified into finite-measurement spots and infinite-measurement spots. 
		For the finite-measurement spots, by referring to \Cref{eqn:cmp_lim2}, it is easy to know that their contribution to \Cref{eqn:cmp_log} diminishes when $k\rightarrow \infty$.
		Therefore, proof using \Cref{eqn:cmp2} can be reduced to only considering the infinite-measurement spots and the rest of the proof is similar to that of \Cref{thm:\proto-dbf-sta-ugv}.
		
%		\hfill\qedsymbol
	\end{proof}
			
%	\begin{rem}
%		The assumption of unbiased sensors are important for the consistency of the estimator. In fact, with unknown non-zero bias, the distribution of $z^j_l$ differs from $P(z^j_l|\xg)$, which invalidates the derivation in \Cref{eqn:cmp_lim2} and the consistency proof. 
%		This assumption also makes intuitive sense.
%		In the extreme case, if each sensor has a very large unknown measurement offset, then the estimated target position of each sensor (without communicating with other sensors) will be very different from each other's.
%		Therefore, no common target position can be correctly obtained when they fuse measurements.
%	\end{rem}	